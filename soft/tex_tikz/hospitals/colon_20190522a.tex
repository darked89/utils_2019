\documentclass[border=.1in]{standalone}
    
\usepackage[T1]{fontenc}
\usepackage[polish]{babel}
\usepackage[utf8]{inputenc}
    
    \usepackage{graphicx}
    \usepackage{tikz}
    %\usepackage{tikz-feynman}
    \usetikzlibrary{shapes,decorations.text,positioning,arrows,calc}

    \begin{document}

\begin{tikzpicture}
	%\draw (-4,-4) grid (4,4);
	\coordinate (c1) at (-6.43,0);
	\coordinate (c2) at (6.78,0);
	\draw[densely dashdotted] (c1) -- node [sloped,below left] {\small{linia cięcia podłużnego}} (c2); 
	   
	\coordinate (tu) at (2.0, 1.5);
	   
	\draw (0,5) node[cylinder,draw=black,thick,aspect=3.9,
		minimum height=14cm,minimum width=10cm,
		shape border rotate=180,
		cylinder uses custom fill,
		cylinder body fill=blue!30,
	cylinder end  fill=blue!10, opacity=.4,anchor=north]
	(A) {};
	
	%guz etc
	\node [circle,
		draw=black,fill=red,inner sep=0pt,
		minimum size=20pt, 
	label=right:{\small{$PT1$}}] (TU) at (tu) {};
	    	
	\node [circle,draw=black,fill=green,inner sep=0pt,minimum size=10pt,label=above:{\small{$UM1$}}, node distance=1,above = of TU] (UM1) {}; 
	  
	\node [circle,draw=black,fill=green,inner sep=0pt,minimum size=10pt,label=right:{\small{$UM3$}}, node distance=1,below right = of TU] (UM2) {};
	  
	\node [circle,draw=black,fill=green,inner sep=0pt,minimum size=10pt,label=left:{\small{$UM2$}}, node distance=1,below left = of TU] (UM3) {};
	 
	\node [circle,draw=black,fill=green,inner sep=0pt,minimum size=10pt,label=above:{\small{$UM13$}}, node distance=2,above right = of TU] (UM13) {};
	  
	\node [circle,draw=black,fill=green,inner sep=0pt,minimum size=10pt,label=below:{\small{$UM12$}}, node distance=2,below = of TU] (UM12) {};
	  
	\node [circle,draw=black,fill=green,inner sep=0pt,minimum size=10pt,label=above:{\small{$UM11$}}, node distance=2,above left = of TU] (UM11) {};
	   
	%% distance arrows
	\draw[thin,<->] (TU.north) --  node [sloped,above] {\small{$1cm$}} (UM1);
	
	\draw[thin,<->] (TU.south west) --  node [sloped,above] {\small{$1cm$}} (UM3) ;
	 
	\draw[thin,<->] (TU.south east) --  node [sloped,above] {\small{$1cm$}} (UM2) ;
	 
	\draw[thin,<->] (TU.north east) --  node [sloped,above] {\small{$2cm$}} (UM13) ;
	  
	\draw[thin,<->] (TU.south) --  node [sloped,above] {\small{$2cm$}} (UM12) ;
	      
	\draw[thin,<->] (TU.north west) --  node [sloped,above] {\small{$2cm$}} (UM11) ;
	 
	%\draw[thin,<->] (TU.south ) --  node [sloped,above] {\small{$2cm$}} (UM12) ;
	
	% \draw[thin,<->] (TU.north east) --  node [sloped,above] {\small{$2cm$}} (UM13) ;
	
	
	 
	%% end distance arrows
	 
	 
	  
	\node [circle,draw=black,fill=green,inner sep=0pt,minimum size=10pt, label=left:{\small{$UM98$}}] (UM98) at (-3,-3) {};
	  
	\node [circle,draw=black,fill=green,inner sep=0pt,minimum size=10pt, label=left:{\small{$UM99$}}] (UM99) at (-5,2) {};
	  
	    
	   
	%end guz etc
	%% nodes
	\node [ellipse, minimum width=2.3cm, minimum height=0.7cm, draw=black,fill=white, rotate=5]  (NODE1) at (-5,6.4) {   } ;
	    
	\node [ellipse, minimum width=1.7cm, minimum height=0.7cm, draw=black,fill=white, rotate=-5]  (NODE2) at (-1.7,6.4) {    } ;
	    
	\node [ellipse, minimum width=2.4cm, minimum height=0.7cm, draw=black,fill=white, rotate=2]  (NODE3) at (2.5,6.4) {    } ;
	    
	\draw (NODE1.east) to[out=-2,in=-1] (NODE2.west);
	\draw (NODE2.east) to[out=-2,in=-1] (NODE3.west);
	     
	\node [ellipse, minimum width=2.3cm,  minimum height=0.7cm, draw=black,fill=white, rotate=-5]  (NODEB1) at (-4.7,-6.0) {    } ;
	   
	\node [ellipse, minimum width=1.7cm,  minimum height=0.7cm, draw=black,fill=white, rotate=-5]  (NODEB2) at (-1.3,-6.5) {    } ;
	   
	\node [ellipse, minimum width=2.4cm,  minimum height=0.7cm, draw=black,fill=white, rotate=2]  (NODEB3) at (2.8,-6.1) {    } ;  
	     
	\draw (NODEB1.east) to[out=-1,in=-1] (NODEB2.west);
	\draw (NODEB2.east) to[out=-1,in=-1] (NODEB3.west);  
	     
	%position=-60:
	   
	\node [circle, draw=black, fill=red, inner sep=0pt, minimum size=10pt, label=left:{\small{$LN1$}} ] (LN1) at (-4.6, 6.5) {};
	   
	\node [circle, draw=black, fill=red, inner sep=0pt, minimum size=10pt, label=below right:{\small{$LN2$}} ] (LN2) at (-4.0, -6.1) {};
	   
	%%end nodes
	
	
	
	%demo
	% \draw (-8,-8) grid (8,8);
	
\end{tikzpicture}
\end{document}