\documentclass[tikz, border=10pt]{standalone}

\usetikzlibrary{positioning,shapes,arrows,calc}


\begin{document}

%\nodeDistA=10mm
\newdimen\nodeDistB
\nodeDistB=20mm

\tikzset{
    position/.style args={#1:#2 from #3}{
        at=(#3.#1), anchor=#1+180, shift=(#1:#2)
    }
}

\begin{tikzpicture}
%\fontfamily{hvmath}{\fontsize{15}{15}\selectfont
%\tikzset{every node}==[font=\sffamily]
  [
    circle type 1/.style={draw=none,align=center, font=\scriptsize},
  ]
  
  \coordinate (c1) at (0,0);
  \coordinate (c2) at (6,0);
  \coordinate (c3) at (0,-6);
  \coordinate (c4) at (-6,0);
  \coordinate (c5) at (0,6);
  \coordinate (tu) at (2.95, -2.5);
  \coordinate (um1) at ($(tu) - (0,1)$);
  \coordinate (um2) at ($(tu) + (1,0)$);
  
  \newlength{\smallercircle}



  \foreach \i / \j [count=\ino] in {6/'',  0.75/fgdsfg, 0.25/fsgsdgf}
    {
     \pgfmathsetmacro{\colmixer}{mod(10*\ino,100)}%
     \path [draw, fill=blue!\colmixer] (c1) ++(0,-\i) arc (-90:270:\i);
     %\ifnum\ino<7
     %  \setlength{\smallercircle}{\i}
     %  \addtolength{\smallercircle}{-7.5pt}
       %\path [decoration={text along path, text={\j}, text align=center, reverse path}, decorate] (c2) ++(0,-\smallercircle) arc (-90:270:\smallercircle);
    
    }
  \draw[densely dashdotted] (c4) -- ($(c1) - (0.75,0)$); 
  \draw[densely dashdotted] ($(c1) +  (0.75,0)$) -- (c2); 
  \draw[densely dashdotted] (c3) -- ($(c1) - (0,0.75)$); 
  \draw[densely dashdotted] ($(c1) + (0,0.75)$)  -- (c5); 
  
  
  \node[circle,draw=black,fill=red,inner sep=0pt,minimum size=22pt, label=right:{\small{$PT1$}}] (TU) at (tu) {};
  \node [circle,draw=black,fill=green,inner sep=0pt,minimum size=10pt,label=below:{\small{$UM1$}}, node distance=1,below right = of TU] (UM1) {}; 
  \node [circle,draw=black,fill=green,inner sep=0pt,minimum size=10pt,label=above:{\small{$UM2$}}, node distance=1,above left = of TU] (UM2) {};
  
  
  \node [circle,draw=black,fill=green,inner sep=0pt,minimum size=10pt,label=below:{\small{$UM12$}}, node distance=2,below left = of TU] (UM12) {};
  
   \node [circle,draw=black,fill=green,inner sep=0pt,minimum size=10pt,label=above:{\small{$UM13$}}, node distance=2,above right = of TU] (UM13) {};
  \node [circle,draw=black,fill=green,inner sep=0pt,minimum size=10pt,label=above:{\small{$UM11$}}, position=160:{\nodeDistB} from TU] (UM11) {};
  
  \node [circle,draw=black,fill=green,inner sep=0pt,minimum size=10pt, label=left:{\small{$UM97$}}] (UM97) at (-3,-3) {};
  
   \node [circle,draw=black,fill=green,inner sep=0pt,minimum size=10pt, label=left:{\small{$UM100$}}] (UM100) at (-3,2) {};
  
     \node [circle,draw=black,fill=green,inner sep=0pt,minimum size=10pt, label=left:{\small{$UM99$}}] (UM99) at (-2.5,4.5) {};
  
     \node [circle,draw=black,fill=green,inner sep=0pt,minimum size=10pt, label=left:{\small{$UM98$}}] (UM98) at (2.5,3.5) {};
  

   
   \node [ellipse, minimum width=1.8cm, draw=black,fill=white, rotate=25]  (S1) at (2.5,-5.4) {  S  } ;
   
      \node [ellipse, minimum width=1.8cm, draw=black,fill=white, rotate=70]  (S100) at (-5.6,2.1) {  S100  } ;
  %position=-60:
  
   \node [ellipse, minimum width=1.2cm, minimum height=0.50cm, draw=black,fill=white, rotate=-45]  (N1) at (4.5,6) {   } ;
  \node [ellipse, minimum width=1.4cm, minimum height=0.55cm, draw=black,fill=white, rotate=-65]  (N2) at (6,4) {   } ;
  
  \node [ellipse, minimum width=1.5cm, minimum height=0.75cm, draw=black,fill=white, rotate=65]  (N2A) at (8,3.5) {   } ;
  
  \node [ellipse, minimum width=1.5cm, minimum height=0.75cm, draw=black,fill=white, rotate=-85]  (N3) at (6.8,1) {   } ;
  
   \node [ellipse, minimum width=1.5cm, minimum height=0.60cm, draw=black,fill=white, rotate=-85]  (N4) at (7,-1) {   } ;
  
  %\node[circle,draw=black,fill=white,inner sep=0pt,minimum size=10pt,label=below:{UM1}] (a) at (um1) {};
  %\node[circle,draw=black,fill=white,inner sep=0pt,minimum size=10pt,label=below:{UM2}] (a) at (um2) {};
  
  %\draw[line] (tu) --  node [up] {\small{$1cm$}} (um1) ;
  
  %\draw[-stealth'] (0,0) -- node[sloped, anchor=center, above, text width=2.0cm] { Given a timeslice} (UM98);
  
  \draw[thin,<->] (TU.south east) --  node [sloped,above] {\small{$1cm$}} (UM1);
  %{\small{$1cm$}} (UM1) ;
  \draw[thin,<->] (TU.north west) --  node [sloped,above] {\small{$1cm$}} (UM2) ;
  \draw[thin,<->] (TU.south west) --  node [sloped,above] {\small{$2cm$}} (UM12) ;
  \draw[thin,<->] (TU.north east) --  node [sloped,above] {\small{$2cm$}} (UM13) ;
  \draw[thin,<->] (TU.west) --  node [sloped,above] {\small{$2cm$}} (UM11) ;
  
  \coordinate (lumphnode) at (7,2.5);
  
  \draw (N1.east) to[out=-2,in=-7] (N2.west);
  \draw (N2.east) to[out=2,in=-5] (lumphnode);
  \draw (N2A.west) to[out=2,in=-5] (lumphnode);
  \draw (N3.west) to[out=2,in=-5] (lumphnode);
  
  \draw (N3.east) to[out=2,in=-7] (N4.west);
  \draw (N4.east) to[out=0.1,in=-0.1] (6.5,-3.5);
  
  \node [circle,draw=black,fill=red,inner sep=0pt,minimum size=10pt,label=left:{\small{$LN2$}} ] (LN2)at (6,4) {};
  
  \node [circle,draw=black,fill=red,inner sep=0pt,minimum size=10pt,label=left:{\small{$LN1$}} ] (LN1) at (7,-1) {};
  
  %loosely dotted
  %\draw (-8,-8) grid (10,10);
  %%\draw [-Implies, line width=3pt] ($(c1)!1/3!(c2)$) -- ($(c1)!2/3!(c2)$);

\end{tikzpicture}
\end{document}
