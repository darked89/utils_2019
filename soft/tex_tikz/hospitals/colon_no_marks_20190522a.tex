\documentclass[border=.1in]{standalone}
    
\usepackage[T1]{fontenc}
\usepackage[polish]{babel}
\usepackage[utf8]{inputenc}
    
    \usepackage{graphicx}
    \usepackage{tikz}
    %\usepackage{tikz-feynman}
    \usetikzlibrary{shapes,decorations.text,positioning,arrows,calc}

    \begin{document}

\begin{tikzpicture}
	%\draw (-4,-4) grid (4,4);
	\coordinate (c1) at (-6.43,0);
	\coordinate (c2) at (6.78,0);
	\draw[densely dashdotted] (c1) -- node [sloped,below left] {\small{linia cięcia podłużnego}} (c2); 
	   
	\coordinate (tu) at (2.0, 1.5);
	   
	\draw (0,5) node[cylinder,draw=black,thick,aspect=3.9,
		minimum height=14cm,minimum width=10cm,
		shape border rotate=180,
		cylinder uses custom fill,
		cylinder body fill=blue!30,
	cylinder end  fill=blue!10, opacity=.4,anchor=north]
	(A) {};
	
	%\draw[<-] (-9,6.5) --  node [left, sloped,above] {\large{$colon\ start$}} (-6.5,6.5) ;
	%guz etc 
	
	\draw[<-] (-9,6.5) --  node [left, sloped,above] {\large{początek\ jelita}} (-6.0,6.5) ;

   \draw[->] (6.0,6.5) --  node [left, sloped,above] {\large{koniec\ jelita}} (9,6.5) ;
	%guz etc 
	  

	  
	    
	   
	%end guz etc
	%% nodes

	
	   
	%%end nodes
	
	
	
	%demo
	%\draw (-10,-10) grid (10,8);
	
\end{tikzpicture}
\end{document}