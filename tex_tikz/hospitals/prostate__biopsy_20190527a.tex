\documentclass[border=.1in]{standalone}
    
\usepackage[T1]{fontenc}
\usepackage[polish]{babel}
\usepackage[utf8]{inputenc}
    
    \usepackage{graphicx}
    \usepackage{tikz}
    %\usepackage{tikz-feynman}
    \usetikzlibrary{shapes,decorations.text,positioning,arrows,calc}

    \begin{document}

\begin{tikzpicture}
	%\draw (-4,-4) grid (4,4);
	

	
	\coordinate (c1) at (-6.43,-2);
	\coordinate (c2) at (6.78,-2);
	%\draw[densely dashdotted] (c1) -- node [sloped,below left] {\small{linia cięcia podłużnego}} (c2); 
	
	%\draw[densely dashdotted] (c1) -- (c2); 
	   
	\coordinate (tu) at (2.0, 1.5);
	   
	\draw (0,0) node[cylinder,draw=black,thick,aspect=10.0,
		minimum height=10cm,minimum width=4cm,
		shape border rotate=180,
		cylinder uses custom fill,
		cylinder body fill=blue!40,opacity=.1, 
	cylinder end  fill=blue!10, opacity=.4,anchor=north]
	(A) {};
	

	

    \node (box1) at  (-7.5,2.0) [align=left] {\huge średnica\ 8mm} ;
    \draw[densely dotted, ->]
    (box1) [bend right=90]  to  (-5.7, -1.0);
    
    \draw[very thick, color=red, <->] (-5.6, -1.0) -- (-3.6, -3.0);
    \draw[very thick, color=red] (-3.6, -3.0) -- (4, -3.0);
    \draw[densely dotted, color=red, <->] (4.0, -3.0) -- (2.0, -1.0);
    
    \draw[densely dotted, color=red] (-5.6, -1.0) -- (2.0, -1.0);
    
    %\draw[very thick, color=green] (3.17, -4.0) -- (3.17, 0.0);
	
	 %\draw [dashed,] (1.0,-4.0) arc[x radius=2.0, y radius=1/3, start angle=90, end angle=270];
	
	
	\draw [densely dotted,] (3.17, -4.0)  [bend left=90] to  (3.17, 0.0);
	%% nodes

	
	   
	%%end nodes
	
	
	
	%demo
	%\draw (-10,-10) grid (10,8);
	
\end{tikzpicture}
\end{document}